\documentclass[14pt]{beamer}
\usepackage[T2A]{fontenc}
\usepackage[utf8]{inputenc}
\usepackage[english]{babel}
\usepackage{amssymb,amsfonts,amsmath,mathtext}
\usepackage{cite,enumerate,float,indentfirst}
\usepackage{graphicx}

\renewcommand{\vec}[1]{\ensuremath{\boldsymbol{#1}}}

\graphicspath{{images/}}

\usetheme{Pittsburgh}
\usecolortheme{whale}

\setbeamercolor{footline}{fg=blue}
\setbeamertemplate{footline}{
  \leavevmode%
  \hbox{%
  \begin{beamercolorbox}[wd=.333333\paperwidth,ht=2.25ex,dp=1ex,center]{}%
    Boris Kudryashov, ITMO University
  \end{beamercolorbox}%
  \begin{beamercolorbox}[wd=.333333\paperwidth,ht=2.25ex,dp=1ex,center]{}%
    St. Petersburg, 2016
  \end{beamercolorbox}%
  \begin{beamercolorbox}[wd=.333333\paperwidth,ht=2.25ex,dp=1ex,right]{}%
  Page \insertframenumber{} of \inserttotalframenumber \hspace*{2ex}
  \end{beamercolorbox}}%
  \vskip0pt%
}

\newcommand{\itemi}{\item[\checkmark]}

\title{\small{Information Theory. 8th Chapter Problems}}
\author{\huge{
Boris Kudryashov \\
\vspace{30pt}
ITMO University
}}


\begin{document}

\maketitle


\begin{frame}
\frametitle{Problems}
\begin{enumerate}
% \footnotesize {
% \small{

    \item[1]
    Derive formulas for performing uniform scalar quantization at a predetermined pitch $\Delta$ and calculating approximating value by quantum number.

    \pause
    \item[2] 
    Calculate the normalized second moment of the $n$-dimensional cubic lattice.
  
    \pause
    \item[3]
    Prove that lattice $A_1$ coincides with the lattice of integers.

\end{enumerate}
\end{frame}


\end{document} 