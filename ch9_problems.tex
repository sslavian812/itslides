\documentclass[14pt]{beamer}
\usepackage[T2A]{fontenc}
\usepackage[utf8]{inputenc}
\usepackage[english]{babel}
\usepackage{amssymb,amsfonts,amsmath,mathtext}
\usepackage{cite,enumerate,float,indentfirst}
\usepackage{graphicx}

\renewcommand{\vec}[1]{\ensuremath{\boldsymbol{#1}}}

\graphicspath{{images/}}

\usetheme{Pittsburgh}
\usecolortheme{whale}

\setbeamercolor{footline}{fg=blue}
\setbeamertemplate{footline}{
  \leavevmode%
  \hbox{%
  \begin{beamercolorbox}[wd=.333333\paperwidth,ht=2.25ex,dp=1ex,center]{}%
    Boris Kudryashov, ITMO University
  \end{beamercolorbox}%
  \begin{beamercolorbox}[wd=.333333\paperwidth,ht=2.25ex,dp=1ex,center]{}%
    St. Petersburg, 2016
  \end{beamercolorbox}%
  \begin{beamercolorbox}[wd=.333333\paperwidth,ht=2.25ex,dp=1ex,right]{}%
  Page \insertframenumber{} of \inserttotalframenumber \hspace*{2ex}
  \end{beamercolorbox}}%
  \vskip0pt%
}

\newcommand{\itemi}{\item[\checkmark]}

\title{\small{Information Theory. 9th Chapter Problems}}
\author{\huge{
Boris Kudryashov \\
\vspace{30pt}
ITMO University
}}


\begin{document}

\maketitle


\begin{frame}
\frametitle{Problems}
\begin{enumerate}
% \footnotesize {
\small{


    \item[1] 
    Assume that the binary inputs of the discrete-time channel without memory with additive Gaussian noise with zero mean and variance $N_0/2$ are values of binary set $\{-\sqrt E,\sqrt E\}$. 
    These signals correspond to $0$ and $1$.
    Hard decisions are made on the output of the channel: 0, if the output signal $y>0$, and $1$ otherwise. Thus, we have Discrete Stationary Channel (DSC).
    Draw a plot of the throughput of this channel as a signal-to-noise ratio function $ E / N_0 $. Compare it with throughput of an original channel (without signal quantization at the input and output).
}
\end{enumerate}
\end{frame}

\begin{frame}
\frametitle{Problems}
\begin{enumerate}
% \footnotesize {
\small{

    \item[2]
    Consider $T>0$, which is called \emph{erasure threshold}.
    Let decisions be made according to the rule:
    \[
    \hat{x}=\left\{\begin{array}{ll}
                0, & y<-T; \\
                {\mbox {erasure}}, & -T\le y\le T; \\
                1, & y>T.
              \end{array}
              \right.
    \]
    
    Obtain an expression for  $T$, which allows to achieve a maximum throughput of corresponding Discrete Channel with Erasure.
    
    Draw a plot of throughput of this channel as a function of signal-to-noise ratio. Compare throughput of this channel and throughput of Gaussian channel.
}
\end{enumerate}
\end{frame}

\begin{frame}
\frametitle{Problems}
\begin{enumerate}
% \footnotesize {
% \small{

    \item[3] 
    Write an expression for the throughput of channel with binary input and continuous output. 
    Using numerical integration, draw a plot of dependence between  throughput of binary half-continuous and signal-to-noise ratio.
    
    Compare these results with the results for the
    Discrete Stationary channel, Discrete Channel with Erasure and Continuous Input Channel.


\end{enumerate}
\end{frame}


\end{document} 