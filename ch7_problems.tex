\documentclass[14pt]{beamer}
\usepackage[T2A]{fontenc}
\usepackage[utf8]{inputenc}
\usepackage[english]{babel}
\usepackage{amssymb,amsfonts,amsmath,mathtext}
\usepackage{cite,enumerate,float,indentfirst}
\usepackage{graphicx}

\renewcommand{\vec}[1]{\ensuremath{\boldsymbol{#1}}}

\graphicspath{{images/}}

\usetheme{Pittsburgh}
\usecolortheme{whale}

\setbeamercolor{footline}{fg=blue}
\setbeamertemplate{footline}{
  \leavevmode%
  \hbox{%
  \begin{beamercolorbox}[wd=.333333\paperwidth,ht=2.25ex,dp=1ex,center]{}%
    Boris Kudryashov, ITMO University
  \end{beamercolorbox}%
  \begin{beamercolorbox}[wd=.333333\paperwidth,ht=2.25ex,dp=1ex,center]{}%
    St. Petersburg, 2016
  \end{beamercolorbox}%
  \begin{beamercolorbox}[wd=.333333\paperwidth,ht=2.25ex,dp=1ex,right]{}%
  Page \insertframenumber{} of \inserttotalframenumber \hspace*{2ex}
  \end{beamercolorbox}}%
  \vskip0pt%
}

\newcommand{\itemi}{\item[\checkmark]}

\title{\small{Information Theory. 7th Chapter Problems}}
\author{\huge{
Boris Kudryashov \\
\vspace{30pt}
ITMO University
}}


\begin{document}

\maketitle


\begin{frame}
\frametitle{Problems}
\begin{enumerate}
% \footnotesize {
% \small{

    \item[1] Prove $H(D)\ge 0$.
    
    \pause
    \item[2] Prove convexity of function rate-distortion for stationary source without use of memoryless property.
    
    \pause
    \item[3] \label{max_dist}
    For a stationary source with memory, prove more precise estimation of $D_0$ in this property:
    \small{
    For arbitrary stationary source $H(D)= 0$ holds
    \begin{equation} \label{maxD}
    D \ge D_0 =\min_y\int_X f(x) d(y,x) dx
    \end{equation}
    }

\end{enumerate}
\end{frame}

\begin{frame}
\frametitle{Problems}
\begin{enumerate}
% \footnotesize {
% \small{

    
    \item[4] \label{orthogonal_tr} Prove, that orthonormal transformation preserves the distance between vectors.
    
    \pause
    \emph{hint}. Prove, that transformation preserves the norm. Square of norm of vector $\vec x$ can be written as $\|\vec x\|^2=\vec x \vec x^T$. Substitute into this statement instead of $\vec x$ vector $\vec y=\vec x A$, where $A$ -- orthonormal transformation matrix.


\end{enumerate}
\end{frame}

\begin{frame}
\frametitle{Problems}
\begin{enumerate}
% \footnotesize {
\small{

    \item[5]
    Find conditional distribution $\varphi(y|x)$, which reaches rate-distortion function of binary source.
    
    \pause
    \item[6]
    Find conditional distribution $\varphi(y|x)$, which reaches rate-distortion function of Gaussian source.
    
    \pause
    \item[7]
    As the set of codes for binary source, binary codes with parity check can be used. Even-weight sequences are approximating code words.
    Depending on the code length, different coding rates and distortions are obtained obtained.
    For Binary Source uniformly distributed characters draw a plot of rate-distortion function $R(D)$ for such a coding method and compare with theoretical function $H(D)$.
}
\end{enumerate}
\end{frame}

\begin{frame}
\frametitle{Problems}
\begin{enumerate}
% \footnotesize {
\small{

    \item[8] \label{gain_db}
    In analog sources coding, the quantisation quality is measured in \emph{signal-to-noise}, expressed \emph{in decibels}, calculated as
    
    \[
    10 \log_{10}\frac{\sigma^2}{D}\quad \mbox{ dB, }
    \]
    where $\sigma^2$ and $D$ -- dispersion and mean square error respectively.
    
    Draw a plot of dependence between the maximum achievable signal-to-noise ratio and a Gaussian source rate. 
    What will be the expected gain in dB of the speed increase by 1 bit per sample at different coding rates?
}
\end{enumerate}
\end{frame}


\begin{frame}
\frametitle{Problems}
\begin{enumerate}
% \footnotesize {
% \small{

    \item[9]
    Use Bleyhut algorithm for finding a
    rate-distortion function for a Binary Source
    without memory. Compare this result with
    obtained analytically.
    
    \pause
    \item[10]
    Use Bleyhut algorithm for finding a
    rate-distortion function for a Gaussian Source
    without memory. Compare this result with
    obtained analytically.

\end{enumerate}
\end{frame}


\end{document} 